% Revision information
% $Author: kaestner $
% $Date: 2010-03-11 14:48:38 +0100 (Thu, 11 Mar 2010) $
% $Revision: 559 $
%---------------------------------------
\PassOptionsToPackage{pdftex}{graphicx}
\documentclass{beamer}
\usetheme{Singapore}
\usepackage{multimedia}
\graphicspath{{figures/}}

\newcommand{\svnrev}{$Rev: 559 $}
\newcommand\ud{\mathrm{d}}
\newcommand\cj{\mathrm{j}}
\newcommand\ev{\mathbf{E}}
\newcommand\sinc{\mathrm{sinc}}
\newcommand\var[1]{\mathrm{var}\left[#1\right]}
\newcommand\E[1]{\mathrm{E}\left[#1\right]}
\newcommand\dint{\mathop{\int\!\!\!\int}}
\newcommand\hyptest[2]{\stackrel{\mathop{\stackrel{#1}{\gtrless}}}{_{_{#2}}}}
\newcommand\hyp[1]{\mathcal{H}_{#1}}
\newcommand\half{\frac{1}{2}}
\newcommand\mat[1]{\mbox{\boldmath$\mathrm{#1}$\unboldmath}}
\newcommand\vect[1]{\mbox{\boldmath$#1$\unboldmath}}
\newcommand\T{^{\mathrm{T}}}
\newcommand\etal{\emph{et~al.}}
\renewcommand\mat[1]{\mbox{\boldmath$\mathrm{#1}$\unboldmath}}
\newcommand\id{\mathbf{id}}
\newcommand\eqgraphics[2]{\begin{minipage}{#1}\includegraphics[width=\textwidth]{#2}\end{minipage}}
\newcommand{\sign}{\mathrm{sign}}
%\renewcommand\hyptest[2]{\stackrel{\mathop{\stackrel{#1}{\gtrless}}}{_{_{#2}}}}
\renewcommand\exp[1]{e^{#1}}
\renewcommand{\deg}{$^{\circ}$}
\newtheorem{criterion}{Criterion}

\title{Using the golden section for dynamic CT}
\author{A. Kaestner}
\institute{NIAG/ASQ, Paul Scherrer Institut, Switzerland}


\begin{document}

\frame{\titlepage}
\setbeamertemplate{footline}
{%
\begin{beamercolorbox}{section in head/foot}
\vskip-3.0mm\includegraphics[height=4mm]{PSI-Logo.pdf}\hskip2mm\insertpagenumber(\insertpresentationendpage)\vfill
\end{beamercolorbox}
}

\title{\includegraphics[height=3.5mm]{PSI-Logo.pdf}\hspace{2mm}Golden section CT}

\frame{
\frametitle{CT of dynamic processes: The problem}
Dynamic processes are hard to observe with CT
\begin{itemize}
 \item CT needs long scan times.
 \item If the interfaces move more than 1 pixel during the scan motion artifacts will appear.
\end{itemize}
\begin{center}
\includegraphics[width=0.4\textwidth]{figures/seq_0351.png}\\
Sequential acquisition
\end{center}
}

\frame{\frametitle{The solution}
\begin{itemize}
 \item Increment the acquisition angle by the Golden section $\phi=\frac{1+\sqrt{5}}{2}$
 \item The sample will always be observed at nearly orthogonal angles.
\end{itemize}
\vskip5pt
\begin{center}
\includegraphics[width=0.4\textwidth]{../../../Papers/SpatioTemporal/trunk/figures/golden26.pdf}
\end{center}

This concept was introduced by \cite{koehler2004_goldensection}
}

\frame{\frametitle{Experiment at ICON}
\begin{itemize}
 \item To investigate the feasibility of the scanning scheme proposed by K\"{o}hler.
 \item Samples: Saturated hyper-accumulator spheres (P. Trtik).
 \item Process: Evaporation during 2h
 \item Change rate: approx 1 pixel / 10 projections
 \item Micro setup at ICON
 \begin{itemize}
    \item Andor DV436 (2048 pixels)
    \item Exposure time: 20s (ROI, +6s read-out)
    \item 512 Projections
    \item Gadox scintillator
 \end{itemize}
\end{itemize}
}

\frame{\frametitle{Result using all data}
\begin{columns}
 \begin{column}{0.5\textwidth}
  \centering
  \includegraphics[width=\textwidth]{figures/seq_0351.png}\\
  Sequential acquisition
 \end{column}
 \begin{column}{0.5\textwidth}
  \centering
  \includegraphics[width=\textwidth]{figures/golden_0350.png}\\
  Golden section acquisition
 \end{column}
\end{columns}
}

\frame{\frametitle{Reconstructing in time and space}
Data set of 924 projection acquired during 8 hours
\begin{center}
\includegraphics[width=0.8\textwidth]{../../../Papers/SpatioTemporal/trunk/figures/timespace_series.pdf}
\end{center}
}

\frame{\frametitle{Rendering the time series}
\begin{tabular}{ccc}
\includegraphics[width=0.3\textwidth]{../../../Papers/SpatioTemporal/trunk/figures/d08_01.png} &
\includegraphics[width=0.3\textwidth]{../../../Papers/SpatioTemporal/trunk/figures/d08_02.png} &
\includegraphics[width=0.3\textwidth]{../../../Papers/SpatioTemporal/trunk/figures/d08_03.png} \\
 62 min & 125 min & 187 min\\
\includegraphics[width=0.3\textwidth]{../../../Papers/SpatioTemporal/trunk/figures/d08_04.png} &
\includegraphics[width=0.3\textwidth]{../../../Papers/SpatioTemporal/trunk/figures/d08_05.png} &
\includegraphics[width=0.3\textwidth]{../../../Papers/SpatioTemporal/trunk/figures/d08_06.png} \\
250 min & 312 min & 375 min
\end{tabular}
}

\frame{\frametitle{Evaluating the angle distribution}
Computing the entropy of 50 realizations of
\[
\Theta_i^N=\{ \mathrm{mod}((i\,N +j)\phi \pi,\pi)\,|\,j \in [0, N-1], j\in \mathbb{Z}^+ \}
\label{eq_goldenrealizations}
\]
\begin{center}
\includegraphics[width=0.5\textwidth]{../../../Papers/SpatioTemporal/trunk/figures/entropy_plot.pdf}
\end{center}

The local minima are found at \framebox{$N_p= N_0 \phi^p$}
}

\frame{\frametitle{Comparing angle distributions}
\begin{columns}
\begin{column}{0.5\textwidth}
\centering
\includegraphics[width=\textwidth]{../../../Papers/SpatioTemporal/trunk/figures/golden26.pdf}\\
Local maximum: 26 Projections
\end{column}
\begin{column}{0.5\textwidth}
\centering
\includegraphics[width=\textwidth]{../../../Papers/SpatioTemporal/trunk/figures/golden34.pdf}\\
Local minimum: 34 Projections
\end{column}
\end{columns}
\vskip10pt
\begin{center}
\framebox{Local minima give almost equidistant acquisition angles.}
\end{center}
}

\frame{\frametitle{Conclusions}
\begin{itemize}
\item Golden section scans are feasible for NI.
\item It is recommended for any dynamic sample in the future.
\item The homogeneity of the angle distribution varies. 
\end{itemize}
}

\frame[allowframebreaks]{\frametitle{References}
\tiny
\bibliographystyle{apalike}
\bibliography{../../../references}
}

\end{document}
