\documentclass[a4paper,11pt]{scrreprt}
\usepackage[utf8x]{inputenc}
\usepackage[pdftex]{graphicx}
\usepackage{geometry}
\usepackage{times}

\geometry{hmargin=2.5cm,vmargin=2.5cm}

% Title Page
\title{Quantitative neutron imaging for modelling of root water uptake in soil}
\author{A. Kaestner and A. Carminati}

\begin{document}
\maketitle

\chapter{Summary}
Some more blah, blah about the roots.

The project is a multi-disiplinary collaboration between the research topics soil hydrology 
and neutron imaging. The aim of the project is to better understand the root water uptake process.
Earlier work has shown that neutron imaging is the most suited method to monitor the water movements 
in soil in two or three dimensions. 

The proposed project is divided into two subtasks which is are assigned one PhD student each. 
One student stationed at the department of Agriculture at the university of G\"{o}ttingen, Germany, 
will focus on the modelling of soil--root interation and the verification using traditional method. 
The second student stationed at Paul Scherrer Institut, Switzerland, will focus on refinements of the
neutron imaging and image processing procedures required for the imaging based verification of the 
water uptake models.

\chapter{Research plan}
\section{State of the art in the field}
\subsection{Root-soil interactions}
\subsubsection{Root water uptake}
\begin{itemize}
\item Importance and exsiting concepts of water flow from soil to roots.
\item Mechanical properties and behavior of roots growing in soil
\end{itemize}
\subsubsection{Rhizospere}
\begin{itemize}
\item Importance of the root-soil interface. 
\item The role/importance of mucilage. 
\item Existing open questions. 
\item Gaps between soil and roots.
\item Cluster roots visible and roots hairs too small.
\end{itemize}
\subsection{Imaging of root systems}
Introduction: Existing methods to observe roots in soil.
The 
\subsubsection{X-ray imaging}
Mirco CT tomcat

\subsubsection{MRI}

\subsubsection{Neutron imaging}
Neutron imaging is a method complementary to X-ray imaging. In first approximation it is 
based on the universal attenuation law similar to X-ray imaging. The mechanism for attenuation differs
however, since neutrons interact with the nucleus of the atom. Often the intensity is reduced
in the sample by scattering or a mix of scattering and absorption.

Methodologically, there are three main fields investigated in neutron imaging; 1
-- neutron-matter interaction in the spatial domain, 2 -- the wave properties of 
the neutron beam,  and 3 -- refinement of the beamline instrumentation.

Computed tomography is a well-known imaging method and the development has
mainly be driven by the medical imaging community using X-ray based systems. The principle
of neutron tomography has much in common with X-ray tomography since the general
attenuation law applies to neutron imaging too. Therefore, neutron CT has also 
become a standard acquisition method at all advanced neutron imaging facilities 
world-wide. For many applications the collimation ratio is sufficiently large to assume
parallel beam geometry for the tomographic reconstruction. There are however
exceptions when large samples are acquired with high resolution detectors. In
these cases it has proven to be beneficial to use reconstruction methods for
cone beam geometry instead of parallel geometry \cite{kaestner2012_wcndt}. 

The restrictions can be caused by the sample dimensions or
the sample infrastructure. The shape of the sample container is restricted by the root 
physiology as well the neutron transmission through the sample. The roots must be allowed 
grow freely in at least two directions while the sample thickness in the beam direction 
must be kept small to allow the neutrons to penetrate the sample. These constraints result in 
thin and wide sample container. A full CT scan of such a container would require that the sample
is to remotely placed from the detector to provide the resolution required for root studies. Here, 
the resolution is limited by the geometric unsharpness. The mathematics of computed 
tomography has evolved the last decades. This development has lead to reconstruction 
methods that produce good result even though the projection data set is incomplete in some sense. A case that is
of special interest for neutron imaging of root soil samples is limited angle tomography. 
This case occurs when samples are hindered to perform a rotation that provides the
complete data set. Another example where limited angle CT is relevant is for fuel cell research where 
the rotation is constrained by cabeling and tubing \cite{lange2011_directt_fuelcell}. 

The limited view problem is severely ill-posed and the number of singular values are proportional
to the size of the missing angular range. It is even exponential if the range of projections is 
very small \cite{natterer2001}. In general, the development of reconstruction methods for 
the limited angle problem has favored iterative reconstruction methods. When methods based on a 
filtered back projection approach the missing data is extrapolated from the existing projection 
data \cite{Ramm1991_limitedangle,ramm1992d}. A review of methods to reconstruct incomplete data is 
given by \cite{clackdoyle2010_incompletedatact}. They propose to perform the filtering step using the
Hilbert transform instead of the ramp filter to handle the instability caused by the missing data.

\subsubsection{Motivation to use neutrons}
Neutron imaging is choosen since the water has a very high contrast to the soil matrix. This can 
also be achieved using NMR but the spatial resolution is less good. Neutron imaging also has the 
advantage that D$_{mathrm{2}}$O delivers a strong contrast against the normal water. The mixing 
takes place without changing the volume of the mixture \cite{longsworth1937_h2o_d2o}. The hydralic properties of the mixture
changes less than when a tracer based on a iodine salt is used. 

\subsection{Image processing of root images}
\begin{itemize}
\item 2D images
\item Time series of 2D images
\item 3D images
\end{itemize}

Roots are imaged booth in two and three dimensions. In many cases root investigations the roots 
are extracted from the soil and placed on a flatbed scanner. The resulting images are the analyzed
using the commercial software WinRhizo (Regent Instruments Inc.). This tool can extract many parameters 
about the root morphology. This does however require high contrast and uniform illumination 
for the segmentation. These conditions cannot be fulfilled by image orginating from a neutron 
imaging experiment. 


\subsection{Applications of root images}
The interface bringing the the information back to the root related questions.

X-rays: Root distribution, roots soil contact, roots water uptake derived from changing water content.

NMR: Root distribution, water content, indirect uptake, tracers, resolution bad.

Neutrons: Root distribution, water content, indirect root water uptake, water fluxes

\cite{matsushima2009_plant}

\section{Current state of own research}

\subsection{Root--soil interactions}
\begin{itemize}
\item Gaps and mucilage (4-6 papers)
\item Heavy water (1 Submitted) figures from Mohsen showing feasibility and model and results. Need for further investigation.
\end{itemize}

\subsection{Imaging}
\subsubsection{Beamlines}
At Paul Scherrer Institut there are two beamlines for neutron imaging. The main
proposer is responsible for the cold neutron imaging beam line ICON. The current
status of this beamline is described in \cite{kaestner2011_ICON}. The beamline provides equipment 
to handle fields of view from 30$\times$30~mm\textsuperscript{2} to 250$\times$250~mm\textsuperscript{2} 
with resolutions up to 26~lp/mm. This is sufficient for most applications with root soil interaction. 
Currently, an initiative with the aim to increase the resolution by a factor 3--6 is in the concept phase.
With this optics it will be possible to study water exchange near the root in closer detail. 

\subsubsection{Computed tomography}
A basic requirement for the investigations is the access to the source code of a
reconstruction software. This has been developed by the main proposer \cite{kaestner2011_muhrec}. 
The CT reconstruction tool has a flexible design that allows exchanging preprossessing 
steps and also to replace the back-projector. The current work has mainly been focussed 
on alternative reconstruction modes for large samples, and the reconstruction of 
spatio-temporal data, and correction methods for the projection data to reduce noise 
and artifacts. In this framework a basic approach to the local tomography problem was 
implemented using tapering of the cropped edges to zero. Normally, when the sample boundary 
is cropped for some interval of observation angles it is to be expected that the
solution shows truncation artifacts and the solution is biased. 

The acquisition times for an nCT are often very long especially for the micro CT. 
The long scan times are a problem if the sample is not at rest since this will 
cause motion artifacts. Our solution to avoid for CT acquisitions of samples 
with an inherent dynamic component with time constants shorter than the scan time
is to use long irregular angluar steps. In this way two projections with adjacent 
angles are separated by a time much longer than the time between two projections. In
\cite{kaestner2011_golden}, two different acquisition protocols and the related
reconstruction procedures are described. An additional important feature of the described
schemes is that subsets of the projection data fullfill the criteria that the
projection data must be well distributed over the whole interval of [0,$\pi$) to be
reconstructed without artifacts. This makes it possible to reconstruct
time-series of volume data with an arbitrary number of time frames, the cost of
higher temporal resolution is lower spatial resolution. The most flexible and
promising strategy is based on acquisition with angle increments based on the Golden
ratio ($\phi=\frac{1+\sqrt{5}}{2}$). Initial experiments using this acquisition 
pattern have already shown good results.

\subsection{Image processing to detect roots in images}
The neutron radiographies of the root samples rarely have homogeneuous intensity
distribution across the sample. These variations are caused by varying material
thickness and composition as well as inhomogeneous distribution of  the moisture content at
different locations in the sample. The local variations renders global thresholding
methods useless. To overcome this, a local thresholding method inspired by \cite{hoover00} using
a mixture image from matched orientation filters has been developed  by the main proposer. 
This method was developed for the analysis carreid out in \cite{menon2007_NetronRoots}. 
Since then, this work this has been the main tool for root segmentation off root sample 
radiographies by users are neutron imaging beam of Paul Scherrer Institut 
(citations to users of root tracker 2D). Currently, this method is very sensitive 
to variations in the sample constitution. 

Three dimensional root structures have different boundary conditions than the
two dimensional image. In \cite{kaestner06a} a method for root segmentation from
X-ray CT images is described. This method is not directly applicable for neutron
images since contrast relations are different. In \cite{kaestner06a} the root
was identified as a connected structure with attenuation coefficients similar to
the pore space of the sample. The root structure in a neutron image is likely to
have a distribution of attenuation coeficients that partly is similar to the moist 
soil matrix. Hence, a global thresholding approach will most likely fail when applied
to a neutron image. 

\section{Detailed research plan}
\subsection{PhD I : Root-soil interaction}
\subsubsection{Work package Ia: xyz}
\subsection{PhD II: Imaging and Image processing}

\subsubsection{Work package IIa: Propose and evaluate root segmentation methods}
Find a segmentation method that is insensitive to the large variations of the image data. 
The variations are represented by both sample composition, type and morphology of 
the root and by local background intensity variations due to the local water content. 
These challenges are posed by both two- and three dimensional image. 

The student will investigate the feasibility of existing methods that a proposed 
in the soil and plant literature and mainly in the medical imaging literature. The 
problem of finding a root in soil in very similar to the problem of finding blood 
vessels in the human body. The currently proposed methods are based on tracking or 
deformable models. The student is supposed to find a method that identifies the 
root structure under conditions with low contrast and high noise levels.

In the project both two and three dimensional data will appear. Therefore, methods 
that can handle both cases must be found. If the proposed method is multi-dimensional 
and capable of handling both two- and three-dimensional data, it would be the most 
elegant solution. Though, multi-dimensionality is not a primary goal of the work package.  

The evalutation of the proposed processing methods shall be done both on synthezied 
numerical images and  on experimental data from previous experiments. The numerical 
models makes it possible to make simulations to find the break-down conditions of 
the proposed method. The experimental data allows to verify how well the numerical 
results relates to an experimental situation. The combination of the results from the 
method evaluation shall make it possible minimize exposure times in the time series 
experiments. 
  
\subsubsection{Work package IIb: Tomo-synthesis}
The most used sample geometry is a thin slab with a rectangular cross section. 
This sample geometry has proven to be ideal for both radiography imaging and roots. 
The thin cross section gives a short line integral through the sample which is good 
for the transmission even at relatively high water contents. The sample can also be 
positioned very close to the detector which reduces the effect of geometrical unsharpness 
in the images. The roots are only constrained in one direction which allows them grow 
relatively naturally in the other two directions. For CT this sample shape is less ideal 
since the rather wide sample requires that the must be positioned so far away from the 
detector that the geometric blurring becomes a problem. The ideal sample shape for CT 
would be a cylinder, but that would possibly lead to detector starvation for wet samples. 
In a cylinder the roots would only be allowed to grow naturally in the vertical direction, 
while they would be constrained in the horizontal direction. The consequence is a unnatural 
growth pattern. 

An additional issue with CT acquistions of the samples is the time needed to complete the scan. 
This too long to be usefull in a dynamic experiment. To meet the experimental boundary conditions 
we propose that the student shall identify and evaluate a suitable reconstruction method for 
tomo-synthesis.  The choice shall be made with the two following work packages in mind. 
The performance in terms of fidelity to a ground truth, singnal to noise ratio, and resolution 
shall be evaluated.   

Water in roots is a dynamic process that mostly is faster than the time needed to acquire a 
CT data set. In this work package the student shall investigate the feasibility of the Golden 
ratio acquisition scheme to acquire first images with less motion artifacts. If this is 
feasible the next step is to use the spatio-temporal feature 
of this acquisition scheme. 

\subsubsection{Work package IIc: Quantitative neutron imaging }
One goal of this proposal is to quantify the water content in the sample. 
One important factor to consider for the water content estimation is eliminate
the effect scattered neutrons. Hydrogen has a large scattering cross-section 
which wil contribute to a biased estimate of the water content. The student 
shall investigate to which extent the scattered neutrons contribute to a 
bias for the relevant sample geometries. In general, a cupped intensity profile 
is expected in the reconstructed images of a scattering sample. R. Hassanein 
developed a method for quantitative neutron imaging in his PhD thesis \cite{hassanein2006diss}. 
The method was develeped for a thermal neutron spectrum and it is needed 
to verify and modify the correction procedure for cold neutrons. 

In addition to the bias caused by the scattering, a bias is expected due to the tomo-synthesis 
reconstruction. This approach works with incomplete data which will introduce an error in the 
reconstructed attenuation coefficients. The student shall investigate the 
impact incomplete projection data has on the accuracy of the reconstructed 
attenuation coefficients. The student shall make both a numerical study and 
demonstrate the performance on experimental data using samples with low scattering cross-section 
as well as wet samples where the water content in the samples is known by gravimetric measurements. 

\subsubsection{Work package IId: Compare perfomance of X-ray vs. Neutron imaging}
Most of the published methods on the topic root-soil imaging work with X-ray based imaging setups. 
As a complement to the main line of this proposal the student shall make study to show how X-ray 
imaging performs under similar conditions. The student shall also investigate the opportunities 
of mixing the two image types. Which additional information can be gained from a mixed apporach. 

The NEUTRA beamline provides the option to use an X-ray source in the same beamline as 
the neutrons. This feature allows pixelwise comparisons between the two imaging modalities. 

The image processing method propsed in work package IIa shall also be applied on the 
X-ray data as an additional stability test of the method.


\subsection{Collaborations PhD I and PhD II}
\subsubsection{Work package I+IIa: Specify experimental boundary conditions}
\subsubsection{Work package I+IIb: Experiments with plants}
\subsubsection{Work package I+IIc: Data evaluation}

\subsection{Visiting student exchange}
Each student should visit the partner laboratory for a short period of time to exchange 
experiences and knowledge.  


\section{Schedule and milestones}
The project is outline in table \ref{tab_milestones}. 
\begin{table}[ht!]
\centering
 \begin{tabular}{|c|c|}
  \hline
  1st year Phase I & \begin{tabular}{c|l} 3 months & General literature study +
intro to NI\\ 9 month & Define numerical and physical phantoms \\ & First
prototype of reconstructor \\ & Experiments planning, first experiments
\end{tabular}\\
\hline
  2nd year Phase II & \begin{tabular}{c|l} 3 months & Literature study
incomplete data\\ & Refinement of prototype \\ 9 months & Filter implementation
and experiments\\ & Joint experiments \end{tabular}\\
\hline
  3nd year Phase II \& III & \begin{tabular}{c|l}6 months & Refining the
algorithms\\ & Joint experiments\\6 months & Write thesis\end{tabular}\\
\hline
 \end{tabular}
\caption{Milestones of the PhD project.}\label{tab_milestones}
\end{table}

\section{Importance and impact}
In general, applications of neutron imaging will gain from the increased quantitative accuracy 
for cold neutron spectra. Quantitative neutron imaging in three dimensions is a topic of great 
interest especially for samples with high hydrogen content. The practical impact of the work of 
PhD II is an improved platform for the root-soil user community visiting the neutron imaging 
beamlines at Paul Scherrer Institut. This will further strengthen Paul Scherrer Institut as a 
world-leading site for neutron imaging. The student will gain knowledge of methods that can be 
applied with X-ray imaging in medical applications.

The applications where the improved imaging setup will have an impact are for imaging studies of
electrochemical processes like fuel cells and lithium-ion batteries. Water transport in soil 
and geology systems in a wider sense than outlined in the proposal. Quantitative imaging 
for applications in nuclear engineering. Examples are samples of hydritized zirconium alloys 
and steady state flow in cooling subchannel systems. 

\bibliographystyle{unsrt}
\bibliography{../references}
\end{document}



\subsection{Investigating the spatio-temporal CT}
Important questions are:
\begin{itemize}
 \item Dose vs Noise for small data sets.
 \item Solution smoothness.
 \item Sensitivity to motion in the sample.
 \item Overlapping subset, redundancy.
 \item Interpretation of the gradients in the motion regions.
\end{itemize}

\section{Old Material}
\subsection{Computed tomography of root samples}
Container 
\subsubsection{Computed tomography with incomplete data}
The sample containers used for 
The sample containers for root system studies are optim 
Clackdoyle and Defrise gives an overview of the latest developments in the
handling of incomplete projection data
sets\cite{clackdoyle2010_incompletedatact}.
\begin{figure}
\caption{General incomplete data illustration.}\label{fig_incomplete}
\end{figure}

In general, the handling of incomplete projection data is solved by filtering
the data using the Hilbert transform instead of the Fourier transform.


\subsubsection{The limited angle tomography}
In some cases it is not possible to rotate the sample the 180$^{\circ}$ that the
required to fulfill data support criterion.
This means that the projection data can not be reconstructed by traditional
means. In medical imaging this is often referred
to tomo-synthesis. When only a fraction of the scan orbit is used it is not
possible to reconstruct the

This is by nature an ill-posed problem and as such the solutions are not unique.
In this project a method to reconstruct limited angle data shall be implemented and verified against
numerical and physical phantoms. The aim is to provide reconstructed vertical slabs of the scanned sample where
the attenuation coefficients are quantifiable. 
\begin{figure}
\caption{Illustration of tomo-synthesis.}\label{fig_tomosynthesis}
\end{figure}

The task is to provide a reconstruction method that can handle the limited angle
problem that is 
optimized for the conditions provided by neutron imaging. The performance of
the 
method shall be verified on both numerical and experimental data. The
verification 
involves image quality in terms of SNR and fidelity to the original data. The
boundaries 
for the reconstructed ROI as a function of the scan angle interval.


\subsection{Samples with inherent dynamics}
Many samples have a dynamic component with time constant shorter than the
acquisition time. 
This will induce motion artifacts in the reconstructed data. The Golden ratio
acquisition scheme
presented in \cite{kaestner2011_golden} has proven successful for
complete data set. 
The student should verify if the method also apropriate for incomplete data
sets. The verification
should be done with main focus on the limited angle tomography since this method
has most 
applications where the sample dynamics may play a role.

\subsection{Experiments with neutrons}
Much of the verification and comparison work of the methods can done on
numerical phantoms. It is still important to verify the methods under actual conditions at several
imagin beamlines. ach beamline has individual charateristics in terms of spectral distribution,
intensity, and instrument hardware. These are factors that needs to be considered in the optimization of the
reconstruction methods. In addition to the instrument conditions, also the applications of the methods
may have an impact on the boundary conditions, e.g. minimum contrast and/or sample geometry. The
initial experiments will be based on scans using test bodies with known geometry and material
composistion. These test bodies shall also be modelled numerically to allow the generation of
pure synthetic data sets for comparison purposes. This will be used to demonstrate and verify
that the. The student will collaborate with people working with the applications of the
newly 
develeped reconstruction methods. There will be two main applications in the
initial phase 
they are soil science in collaboration with professor Andrea Carminati,
University G\"{o}ttingen, Germany,
and fuel cell research with Dr. Pierre Boillat, Paul Scherrer Institut,
Switzerland. 

\begin{itemize}
\item Tomo synthesis -- a review is required
\end{itemize}
Quotes from Natterer\cite{natterer2001}:
\begin{itemize}
\item[Page 145] Thus the number of small singular values of R is proportional to the
size of the missing angular range.
\item[Page 146] On the other hand, if the missing range is large, i.e., $\phi$ close to 0, then
Now the decay is truly exponential, and hence the problem is severely ill-posed. Practical
experience shows that it is quite difficult to get decent reconstructions for 0 $<$ $\pi$/3, i.e.,
from an angular range of less than 120°
\item[Page 147] A general approach to incomplete data problems is data completion. One simply
estimates the data in the missing range, either from information about the object (such as its
boundary) or by mathematical extrapolation into the missing range.
\item[Page 148] Iterative methods can be applied to incomplete data without any preprocessing of the
data. This does not mean that the results are always satisfactory. An important feature of
iterative methods is the possibility to incorporate a priori knowledge about the object to be
reconstructed.
\end{itemize}

